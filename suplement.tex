\documentclass[12pt]{article}
\usepackage{ucs}
\usepackage[utf8x]{inputenc}
\usepackage[russian]{babel}  
\usepackage{amsmath}
\title{Supplement materials}
\date{}
\author{Mysin I.E.}

\begin{document}
	\maketitle
	Pyramidal neurons
	\newline
	Уравнение для сомы, радиатума и ориенса
	\begin{eqnarray}
C\frac{dV_s}{dt} = -I_l - I_{Kdr}-I_{Na} -I_A - I_M-I_H-I_{CaL} - I_{sAHP}-I_{mAHP} -I_{CaR}-I_{buff}-\nonumber \\
-I_{syn}+I_{ext}
	\end{eqnarray}
\newline

Уравнение для аксона
\newline
	\begin{equation}
	C\frac{dV_a}{dt} = -I_l - I_{Kdr}-I_{Na}- I_M - I_{syn}
    \end{equation}
\newline

Уравнение для LM
\newline
\begin{equation}
C\frac{dV_LM}{dt} = -I_l - I_{Kdr}-I_{Na}- I_A - I_{syn} + I_{ext}
\end{equation}
\newline

Натриевый каналы
\newline
\begin{equation}
I_{Na} = g_{max, Na} \cdot m^2 \cdot h\cdot s \cdot(V - E_{Na})
\end{equation}
\newline
For dendiritic compartments
\newline
\begin{equation}
m_{\infty} = \frac{1}{1 + exp(-\frac{V + 40}{3})} \ 
\tau_m = 0.05 ms
\end{equation}
\newline
\begin{equation}
h_{\infty} = \frac{1}{1 + exp(-\frac{V + 45}{3})} \ 
\tau_h = 0.5 ms
\end{equation}
\newline

For soma/axon compartments
\begin{equation}
m_{\infty} = \frac{1}{1 + exp(-\frac{V + 44}{3})} \ 
\tau_m = 0.05 ms
\end{equation}
\newline
\begin{equation}
h_{\infty} = \frac{1}{1 + exp(-\frac{V + 49}{3.5})} \ 
\tau_h = 1 ms
\end{equation}
\newline


For all compartments
\begin{equation}
s_{\infty} = \frac{1 + Na_{att} exp(0.5(V+60))}{1 + exp(0.5(V+60))}
\end{equation}
\newline

\begin{equation}
\tau_s = \frac{0.00333 exp(0.0024(V + 60)Q) }{1 + exp(0.0012(V + 60)Q)}
\end{equation}
\newline

\begin{equation}
Q = \frac{F}{RT}
\end{equation}
\newline


The delayed rectifier current is given by:

\begin{equation}
I_{Kdr} = g_{max, Kdr} \cdot n^2 \cdot (V - E_{K})
\end{equation}
For dendiritic compartments
\begin{equation}
n_{\infty} = \frac{1}{1 + exp(-0.5(V + 42))} \ 
\tau_n = 2.2 ms
\end{equation}
For soma/axon compartments
\begin{equation}
n_{\infty} = \frac{1}{1 + exp(-0.3333(V + 46.3))} \ 
\tau_n = 3.5 ms
\end{equation}
\newline

The fast inactivating A-type K+ current is described by
\begin{equation}
I_{A} = g_{max, A} \cdot n_A \cdot l \cdot (V - E_{K})
\end{equation}

\begin{equation}
\alpha_{n_A} = \frac{-0.01(V + 21.3)}{exp( \frac{V + 21.3}{-35}) - 1}
\end{equation}
\newline
\begin{equation}
\beta_{n_A} = \frac{0.01(V + 21.3)}{exp( \frac{V + 21.3}{35}) - 1}
\end{equation}
\newline

\begin{equation}
\alpha_{l} = \frac{-0.01(V + 58)}{exp( \frac{V + 58}{8.2}) - 1}
\end{equation}
\newline
\begin{equation}
\beta_{l} = \frac{0.01(V + 58)}{exp( \frac{V + 58}{-8.2}) - 1}
\end{equation}
\newline
\begin{equation}
	\begin{matrix}
	\tau_{l} = & =
	& \left\{
	\begin{matrix}
	 5 + 0.26(V + 20) & \mbox{if } V > 20 \\
	 5 & \mbox{otherwise }
	\end{matrix} \right.
	\end{matrix}
\end{equation}
\newline
The hyperpolarizing H-current is given by:
\begin{equation}
I_{H} = g_{max, H} \cdot H \cdot (V - E_{H})
\end{equation}
\begin{equation}
H_{\infty} = \frac{1}{1 + exp(0.125(V + 75))} 
\end{equation}

\begin{equation}
\tau_H = \frac{exp(0.033264 (V + 75))}{0.35(1 + exp(0.0083(V + 75)))}
\end{equation}
\newline

The slowly activating voltage-dependent potassium current, IM, is  given by the equations:
\begin{equation}
I_M = g_{max, M} \cdot T_{adj} \cdot q \cdot (V - E_{K}) \ 
T_{adj} = 10^{-4} \cdot 2.3^{0.1(T - 296)}
\end{equation}
\begin{equation}
\alpha_{q} = \frac{10^{-3}(V+30)}{1-exp(\frac{V+30}{-9})}
\end{equation}
\begin{equation}
\beta_{q} = \frac{-10^{-3}(V+30)}{1-exp(\frac{V+30}{9})}
\end{equation}
\newline

The slow after-hyperpolarizing current, IsAHP
\begin{equation}
I_{sAHP} = g_{max, sAHP} \cdot p^3 \cdot (V - E_{K})
\end{equation}
\begin{equation}
\frac{dp}{dt} = \frac{C_{Ca} - p(1 + C_{Ca})}{\tau_p(1 + C_{Ca})}
\end{equation}
\begin{equation}
\tau_{p} = max( \frac{1}{0.003 \cdot (1 + C_{Ca}) 3^(0.1(T-295))}, \ 0.5)
\end{equation}
\begin{equation}
C_{Ca} = \Bigl(\frac{[Ca^{2+}]_{in}}{0.025} \Bigl)^2
\end{equation}
\newline
The medium after-hyperpolarizing current, I mAHP (Moczydlowski and Latorre, 1983), is given by:
\begin{equation}
I_{mAHP} = g_{max, mAHP} \cdot a \cdot (V - E_{K})
\end{equation}
\begin{equation}
\alpha_{a} = \frac{0.48}{1 + \frac{0.18 \cdot exp(-1.68 V Q)}{[Ca^{2+}]_{in}}}
\end{equation}
\begin{equation}
\beta_{a} = \frac{0.28}{1 + \frac{[Ca^{2+}]_{in}}{0.011 \cdot exp(-2 V Q)}}
\end{equation}
\newline
The somatic high-voltage activated (HVA) L-type Ca2+ current is given by
\begin{equation}
I_{CaL} = g_{max, soma CaL} \cdot b \cdot \frac{0.001}{0.001 + [Ca^{2+}]_{in}} \cdot (V - E_{Ca}) \ !!!???
\end{equation}
\begin{equation}
\alpha_{b} = \frac{-5.055 \ (V + 27.01)}{exp( \frac{(V + 27.01}{-3.8})  - 1}
\end{equation}
\begin{equation}
\beta_{b} = 4.7 \ exp \Bigl( \frac{V + 63.01}{-17} \Bigl)
\end{equation}
whereas the dendritic L-type calcium channels have different kinetics:
\begin{equation}
I_{CaL} = g_{max, dend CaL} \cdot b^3 \cdot c \cdot (V - E_{Ca})
\end{equation}
\begin{equation}
\alpha_{b} = \frac{1}{1 + exp(-(V + 37)) }
\end{equation}
\begin{equation}
\beta_{b} = \frac{1}{1 + exp(2(V + 41)) }
\end{equation}
\begin{equation}
\alpha_{c} = ?
\end{equation}
\begin{equation}
\beta_{c} = ?
\end{equation}
\begin{equation}
\tau_{b} = 3.6
\end{equation}
\begin{equation}
\tau_{c} = 29
\end{equation}
\newline
The low-voltage activated (LVA) T-type Ca2+ channel kinetics are given by:
\begin{equation}
I_{CaT} = g_{max, soma CaT} \cdot d^2 \cdot r \cdot \frac{0.001}{0.001 + [Ca^{2+}]_{in}} \cdot (V - E_{Ca}) \ !!!???
\end{equation}
\begin{equation}
\alpha_{d} = \frac{-0.196(V - 19.88)}{ exp(-0.1(V - 19.88)) - 1}
\end{equation}
\begin{equation}
\beta_{d} = 0.46 exp( -V/22.73 ) 
\end{equation}
\begin{equation}
\alpha_{r} = 0.00016 exp( \frac{V+57}{-19})
\end{equation}
\begin{equation}
\beta_{r} = \frac{1}{1 + exp(-0.1(V-15)) }
\end{equation}
\newline
The HVA R-type Ca2+ current is described by:
\begin{equation}
I_{CaR} = g_{max, CaR} \cdot w^3 \cdot j \cdot  \cdot (V - E_{Ca}) 
\end{equation}
equations for dendritic CaR channels are:
\begin{equation}
w_{\infty} = \frac{1}{ (1 + exp(-0.3333(V+48)) ) } \ \ \tau_{w} = 50 ms
\end{equation}
\begin{equation}
j_{\infty} = \frac{1}{1 + exp(V + 53) } \ \ \tau_{j} = 5 ms
\end{equation}
while for the somatic CaR channels:
\begin{equation}
w_{\infty} = \frac{1}{ (1 + exp(-0.3333(V+60)) ) } \ \ \tau_{w} = 100 ms
\end{equation}
\begin{equation}
j_{\infty} = \frac{1}{1 + exp(V + 62) } \ \ \tau_{j} = 5 ms
\end{equation}
\newline




\end{document}


