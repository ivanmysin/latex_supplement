\documentclass[12pt]{article}
\usepackage{ucs}
\usepackage[utf8x]{inputenc}
\usepackage[russian]{babel}  
\usepackage{amsmath}
\title{Supplement materials}
\date{}
\author{Mysin I.E.}

\begin{document}
	\maketitle
	Pyramidal neurons
	\newline
	Уравнение для сомы, радиатума и ориенса
	\begin{eqnarray}
C\frac{dV_s}{dt} = -I_l - I_{Kdr}-I_{Na} -I_A - I_M-I_H-I_{CaL} - I_{sAHP}-I_{mAHP} -I_{CaR}-I_{buff}-\nonumber \\
-I_{syn}+I_{ext}
	\end{eqnarray}
\newline

Уравнение для аксона
\newline
	\begin{equation}
	C\frac{dV_a}{dt} = -I_l - I_{Kdr}-I_{Na}- I_M - I_{syn}
    \end{equation}
\newline

Уравнение для LM
\newline
\begin{equation}
C\frac{dV_LM}{dt} = -I_l - I_{Kdr}-I_{Na}- I_A - I_{syn} + I_{ext}
\end{equation}
\newline

Натриевый каналы
\newline
\begin{equation}
I_{Na} = g_{max, Na} \cdot m^2 \cdot h\cdot s \cdot(V - E_{Na})
\end{equation}
\newline
For dendiritic compartments
\newline
\begin{equation}
m_{\infty} = \frac{1}{1 + exp(-\frac{V + 40}{3})}
\end{equation}
\newline
\begin{equation}
h_{\infty} = \frac{1}{1 + exp(-\frac{V + 45}{3})}
\end{equation}
\newline

For soma/axon compartments
\begin{equation}
m_{\infty} = \frac{1}{1 + exp(-\frac{V + 44}{3})}
\end{equation}
\newline
\begin{equation}
h_{\infty} = \frac{1}{1 + exp(-\frac{V + 49}{3.5})}
\end{equation}
\newline


For all compartments
\begin{equation}
s_{\infty} = \frac{1 + Na_{att} exp(0.5(V+60))}{1 + exp(0.5(V+60))}
\end{equation}
\newline

\begin{equation}
\tau_s = \frac{0.00333 exp(0.0024(V + 60)Q) }{1 + exp(0.0012(V + 60)Q)}
\end{equation}
\newline

\begin{equation}
Q = \frac{F}{RT}
\end{equation}
\newline


The delayed rectifier current is given by:

\begin{equation}
I_{Kdr} = g_{max, Kdr} \cdot n^2 \cdot (V - E_{K})
\end{equation}
For dendiritic compartments
\begin{equation}
n_{\infty} = \frac{1}{1 + exp(-0.5(V + 42))}
\end{equation}
For soma/axon compartments
\begin{equation}
n_{\infty} = \frac{1}{1 + exp(-0.3333(V + 46.3))}
\end{equation}
\newline



\end{document}


