\documentclass[a4paper,12pt]{article}
\usepackage{ucs}
\usepackage[utf8x]{inputenc}
\usepackage[russian]{babel}  
\usepackage{amsmath}
\title{Supplement materials}
\author{Mysin I.E.}
\date{}

\begin{document}
\maketitle

\section{Neurons models}
\subsection{CA1 Pyramidal cell}
Пирамидные нейроны описывались с помощью 19-ти компартментной модели. В модели принимались во внимание сома, базальные дендриты, проксимальная и дистальная часть апикального дендрита. \par
Уравнение баланса для соматического компартмента:
\begin{eqnarray}
C\frac{dV_s}{dt}=-I_L-I_{Na}-I_{kdr}-I_A-I_M-I_H-I_{sAHP}-I_{mAHP}-I_{CaL}- \nonumber \\ -I_{CaT}-I_{CaR}-I_{buff}-I_{syn} + I_{ext}
\end{eqnarray}

Уравнение баланса для аксона:
\begin{equation}
C\frac{dV_a}{dt}=-I_L-I_{Na}-I_{kdr}-I_M-I_{syn}
\end{equation}

Уравнение баланса для базальных дендритов и проксимальной части апикального дендрита:
\begin{eqnarray}
C\frac{dV_{rad,ori}}{dt} =-I_L-I_{Na}-I_{kdr}-I_A-I_M-I_H-I_{sAHP}-I_{mAHP}- \nonumber \\-I_{CaL}-I_{CaT}-I_{CaR}-I_{buff}-I_{syn}+I_{ext}
\end{eqnarray}
Уравнение баланса дистальной части апикального дендрита:
\begin{equation}
C\frac{dV_{LM}}{dt}=-I_L-I_{Na}-I_{kdr}-I_A-I_{syn}+I_{ext}
\end{equation}
where \textit{I\textsubscript{L}} is the leak current,  \textit{I\textsubscript{Na}} is the fast sodium current, \textit{I\textsubscript{kdr}} is the delayed rectifier potassium current, \textit{I\textsubscript{A}} is the A-type potassium current, \textit{I\textsubscript{M}}
is the M-typepotassium current, \textit{I\textsubscript{H}} is a hyperpolarizing H-type current,
\textit{I\textsubscript{CaL}}, \textit{I\textsubscript{CaT}} and
\textit{I\textsubscript{CaR}} are the L-, T- and R-type \textit{Ca\textsuperscript{2+}} currents, respectively,
\textit{I\textsubscript{sAHP}} and \textit{I\textsubscript{mAHP}} are
slow and medium Ca\textsuperscript{2+} activated K\textsuperscript{+} currents,\textit{ I\textsubscript{buff}}
is a calcium pump/buffering mechanism and \textit{I\textsubscript{syn}} is the synaptic current. \textit{ I\textsubscript{ext}} is the tonic current and noise. The
parameters for all ionic currents are listed in Table \ref{table:ca1_pyramidal_cell_parameters}. \par
The sodium current is described by:
\begin{equation}
I_{Na}= g_{max, Na}\cdot m^2\cdot h\cdot s\cdot (V-E_{Na})
\end{equation}
an variable \textit{s} is introduced to account for dendritic location‑dependent slow attenuation of the sodium current (Poirazzi et al., 2003a, b). Activation and inactivation kinetics for ${I_{Na}}$ are given by:

\begin{equation}
m_{\infty}=\frac {1}{1+exp(-\frac{V+40} {3})} , \ \ \tau_m=0.05 \ ms
\end{equation}

\begin{equation}
h_{\infty}=\frac {1}{1+exp(\frac{V+45}{3})},  \  \tau_h = 0.5 \ ms,
\end{equation}

\begin{equation}
s_{\infty}=\frac{1+Na_{att}\cdot exp(\frac{V+60}{2})}{1+exp(\frac{V+60}{2})}
\end{equation}


\begin{equation}
\tau_s=\frac{0.00333\cdot  exp(0.0024 \cdot (V+60)\cdot
		Q(degC))}{1+exp(0.0012 \cdot (V+60)\cdot Q(degC))}
\end{equation}

The function \textit{Q(T)} is given by:
\begin{equation}
\label{eq:QT}
Q(degC)= \frac{F}{ R \cdot T }
\end{equation}
where \textit{R }=\textit{ }8.315\textit{ joule/degC}, \textit{F }=\textit{ }9.648 $\cdot 10^4 \ Coul$,
\textit{T }=\textit{273.16}~in degrees Kelvin and \textit{degC}~is the temperature in degrees celsius.~variable represents the degree of sodium current
attenuation and varies linearly from soma to distal trunk
Na\textsubscript{att}$\in$~[0, 1]: 1 -maximum 0 - zero attenuation (Table \ref{table:ca1_pyramidal_cell_parameters}). The delayed rectifier current is given by:

\begin{equation}
I_{Kdr} = g_{Kdr} \cdot m^2 \cdot (V-E_K)
\end{equation}

\begin{equation}
m_{\infty}=\frac{1}{1+exp(-\frac{V+42}{2})}, \tau_m = 2.2 \ ms
\end{equation}

The sodium and delayed rectifier channel properties are slightly different in the soma, axis and dendritic arbor. To fit
experimental data regarding the backpropagation of spike trains, soma and axon compartments have a lower threshold for
Na\textsuperscript{+} spike initiation (-57 mV) than dendritic ones -50 mV. Thus, the $m_{\infty}$ and $h_{\infty}$
somatic/axonic HH channel kinetics as well as the time constants for both $I_{Na}^{sa}$ and $I_{Kdr}^{sa}$, are modified as follows. For the sodium

\begin{equation}
m_{\infty}^{sa}=\frac {1}{1+exp(-\frac{V+44}{3})} ,\ \ 
h_{\infty}^{sa}=\frac {1}{1+exp(\frac{V+49}{3.5})}
\end{equation}

while for the potassium delayed rectifier

\begin{equation}
m_{\infty}^{sa}=\frac {1}{1+exp(-\frac{V+46.3}{3})}
\end{equation}

The somatic time constant for somatic/axonic \textit{Na}\textit{\textsuperscript{+}}~channel activation is kept the
same $\tau$\textsubscript{m} = 0.05 ms while for inactivation is set to $\tau$\textsubscript{h}= 1 ms. The $\tau$-value for the delayed rectifier channel activation is set to $\tau_{m}= 3.5 ms$.
The fast inactivating A-type K\textsuperscript{+ }current is described by
\begin{equation}
I_A = g_{max, A} \cdot n \cdot l \cdot (V - E_K)
\end{equation}

\begin{equation}
n_{\infty} = \frac{\frac{-0.01(V+21.3)}{exp(-(V+21.3)/35)-1}}{\frac{-0.01(V+21.3)}{exp(-(V+21.3)/35)-1} + \frac{0.01(V+21.3)}{exp( (V+21.3)/35)-1} } \ \ \tau_n=0.2 \ ms
\end{equation}

\begin{equation}
l_{\infty} = \frac{\frac{-0.01(V+58)}{exp( (V+58)/8.2)-1}}{\frac{-0.01(V+58)}{exp( (V+58)/8.2)-1} + \frac{0.01(V+58)}{exp(-(V+58)/8.2)-1}}
\end{equation}

\begin{equation}
\begin{matrix}
\tau_l & =
& \left\{
\begin{matrix}
5+2.6(V+20)/10 & \mbox{if } V > 20 mV \\
5 & \mbox{otherwise }
\end{matrix} \right.
\end{matrix}
\end{equation}


The hyperpolarizing h-current is given by
\begin{equation}
I_H=g_{max, H} \cdot H \cdot (V-E_H)
\end{equation}

\begin{equation}
H_{\infty }=\frac {1}{1+exp((V-V_{half})/8)}  
\end{equation}

\begin{equation}
\tau_{H}=\frac{exp( 0.003364\cdot
	(V-V_{halft})}{0.05 \cdot q10^{(T-33)/10} \cdot
	(1+a_{H})}
\end{equation}


\begin{equation}
a_{H}=exp( 0.008316 \cdot (V-V_{halft}))
\end{equation}
where \textit{q10} =  0.4, \textit{V\textsubscript{halft}}
see in Table \ref{table:ca1_pyramidal_cell_parameters}.

The slowly activating voltage‑dependent potassium current, \textit{I}\textit{\textsubscript{M}},~is given by the
equations:


\begin{equation}
I_m=10^{-4}\cdot T_{adj}(T)\cdot g_m\cdot m\cdot
(V-E_K)
\end{equation}


\begin{equation}
\alpha_m =  \frac{10^{-3}\cdot(V+30)}{(1-exp(-(V+30)/9))T_{adj}(T)} 
\end{equation}
\begin{equation}
\beta_m = \frac{ -10^{-3}\cdot(V+30)}{(1-exp((V+30)/9))T_{adj}(T)}
\end{equation}
\begin{equation}
T_{adj}(T)=2.3^{(T - 287)/10}
\end{equation}


The slow after-hyperpolarizing current, is given by:
\begin{equation}
I_{sAHP}= g_{max, sAHP} \cdot m^3 \cdot (V-E_K)
\end{equation}

\begin{equation}
\frac{dm}{dt}=\frac{\frac{Cac}{(1+Cac)}-m}{\tau_m}
\end{equation}

\begin{equation}
\tau_m=max \Big(\frac {1}{0.003 \cdot (1+Cac)\cdot 3^{(\text{deg}C-22)/10}}, \ 0.5 \Big)
\end{equation}
where 
$Cac=(40 \cdot [Ca^{2+}]_{in})^2$.

The medium after-hyperpolarizing current, I\textsubscript{mAHP} (Moczydlowski and Latorre, 1983), is given by:

\begin{equation}
I_{mAHP}= g_{max, mAHP}\cdot m\cdot (V-E_K)
\end{equation}

\begin{equation}
\alpha_m(V)=\frac{0.48}{1+\frac{0.18}{[Ca^{2+}]_{in}}\cdot exp(-1.68\cdot V\cdot Q(T))}
\end{equation}

\begin{equation}
\beta_m(V)=\frac{0.28}{1+\frac{[Ca^{2+}]_{in}}{0.011\cdot exp(-2\cdot V\cdot Q(T))}}
\end{equation}

Q(T) see formula (\ref{eq:QT}). \par
The somatic high-voltage activated (HVA) L-type Ca\textsuperscript{2+} current is given by


\begin{equation}
I_{CaL}^s= g_{max, CaL}^s\cdot m\cdot \frac{0.001 \cdot ghk(V, [Ca^{2+}]_{in}, [Ca^{2+}]_{out}) }{0.001 + [Ca^{2+}]_{in}}
\end{equation}

\begin{equation}
\alpha_m(V)= \frac{-0.275 \cdot (V+27.01)}{exp(-(V+27.01)/3.8)-1}
\end{equation}

\begin{equation}
 \beta_m(V)= 4.7 \cdot exp(-(V+63.01)/17)
\end{equation}


whereas the dendritic L-type calcium channels have different kinetics: 


\begin{equation}
I_{CaL}^d= g_{max, CaL}^d \cdot m^3 \cdot h \cdot
(V-E_{Ca})
\end{equation}

\begin{equation}
m_{\infty}(V)=\frac {1}{1+exp(-(V+37))}
\end{equation}
\begin{equation}
h_{\infty}(V)=\frac {1}{1+exp( (V+41)/0.5)}
\end{equation}

Their time constants are equal to $\tau_m$ = 3.6\textit{ms} and $\tau_h$=29\textit{ms}. The low-voltage activated (LVA) T-type Ca\textsuperscript{2+} channel kinetics are given by:
\begin{equation}
I_{CaT} = g_{max, CaT}\cdot m^2\cdot h \cdot \frac{0.001 \cdot ghk(V,[Ca^{2+}]_{in},[Ca^{2+}]_{out})}{0.001+[Ca^{2+}]_{in} }
\end{equation}

\begin{eqnarray}
ghk(V,\ [Ca^{2+}]_{in},\ [Ca^{2+}]_{out}) = \nonumber \\
= -x\cdot (1 - [Ca^{2+}]_{out}/[Ca^{2+}]_{in} \cdot exp(V/x)) \cdot f(V/x)
\end{eqnarray}

\begin{equation}
x=\frac{0.0853\cdot T}{2}, \\
f(z) = \begin{cases} 1-\frac {z}{2}, & if \ |z|<10^{-4} \\ \frac z{e^z-1}, & \  otherwise \ \end{cases}
\end{equation}

\begin{equation}
\alpha_m(V)=-0.196\cdot \frac{(V-19.88)}{exp(-(V-19.88)/10)-1} 
\end{equation}

\begin{equation}
\beta_m(V) = 0.046\cdot exp(-(V/22.73))
\end{equation}

\begin{equation}
\alpha_h(V)= 0.00011\cdot exp(-(V+57)/19)
\end{equation}
\begin{equation}
\beta_h(V)=\frac {0.68}{exp(-(V-15)/10)+1}
\end{equation}
where $[Ca^{2+}]_{in}$ and $[Ca^{2+}]_{out}$ are the internal and external calcium concentrations. The HVA R-type Ca2+ current is described by:
\begin{equation}
I_{CaR} = g_{max, CaR}\cdot m^3\cdot h\cdot (V-E_{Ca})
\end{equation}

The difference between somatic and dendritic CaR currents lies in the $\alpha(V)$, $\beta(V)$ and $\tau$ ~parameter
values. For the somatic current, $\tau_m$ = 100\textit{ms} and
$\tau_h$ = 5\textit{ms} while for the dendritic current $\tau_m$ = 50\textit{ms} and $\tau_h$=5 \textit{ ms}. The $\alpha(V)$ and $\beta(V)$ equations for dendritic CaR channels are:


\begin{equation}
m_{\infty}(V)=\frac {1}{1+exp(-(V+48.5)/3)}
\end{equation}
\begin{equation}
h_{\infty}(V)=\frac {1}{1+exp(V+53)}
\end{equation}
while for the somatic CaR channels:
\begin{equation}
m_{\infty}(V)=\frac {1}{1+exp(-(V+60)/3)}
\end{equation}
\begin{equation}
h_{\infty}(V)=\frac {1}{1+exp(V+62)}
\end{equation}

Finally, a calcium pump/buffering mechanism is inserted at the cell body and along the apical and basal trunk. The
mechanism is taken from~(Destexhe, Mainen, \& Sejnowski, 1994). The factor for
$Ca^{2+}$ entry was changed from
\textit{f\textsubscript{e}} = 10,000~to \textit{f\textsubscript{e}} = 10,000/18~and the rate of calcium removal was made 7 times faster. The kinetic equations are given by:

\begin{equation}
Ca_{input flow} = \begin{cases} \frac{-f_e \cdot I_{Ca, sum} }{0.2 \cdot F}, & if \ Ca_{input flow} > 0  \\ 0, & \  otherwise \ \end{cases}
\end{equation}

\begin{equation}
\frac{d[Ca^{2+}]_{in}}{dt}=Ca_{input flow} +\frac{[Ca^{2+}]_{0}-[Ca^{2+}]_{in}}{\tau_{Ca}}
\end{equation}

$[Ca^{2+}]_{0} = 10^{-4} \ mM, \ \tau_{Ca} = 1400 \ ms$


\bigskip

\subsection{Bistratified cells}
Дендритные нейроны описывались с помомощью 13-ти компартметной модели. Балансовое уравнение для потенциала для всех компартментов:
\begin{eqnarray}
C\frac{dV}{dt}=-I_L-I_{Na}-I_{Kdr}-I_A-I_{CaL}-I_{CaN} - \nonumber \\ -I_{AHP}-I_C-I_{syn}+I_{ext}
\end{eqnarray}
where \textit{C} is the membrane capacitance, \textit{V} is the membrane potential, \textit{I\textsubscript{L}}
is the leak current, \textit{I\textsubscript{Na}} is the sodium current, \textit{I\textsubscript{Kdr}} is the fast delayed rectifier K\textsuperscript{+} current, \textit{I}\textit{\textsubscript{A}} is the A-type K\textsuperscript{+} current,\textit{I\textsubscript{CaL}}
is the L-type Ca\textsuperscript{2+} current, \textit{I}\textit{\textsubscript{CaN}} is the N-type
Ca\textsuperscript{2+} current, \textit{I}\textit{\textsubscript{AHP}} is the Ca\textsuperscript{2+}-dependent
K\textsuperscript{+} (SK) current, \textit{I}\textit{\textsubscript{C}} is the Ca\textsuperscript{2+} and
voltage-dependent K\textsuperscript{+} (BK) current and \textit{I}\textit{\textsubscript{syn}} is the synaptic current.
The conductance and reversal potential values of all ionic currents are listed in Table ... .
The sodium current and its kinetics are described by:
\begin{equation}
I_{Na}=g_{max, Na} \cdot m^3 \cdot h \cdot (V-E_{Na})
\end{equation}

\begin{equation}
\alpha_m(V)=\frac{-0.3\cdot(V-25)}{1-exp(-0.2\cdot(V-25))}, \ \  \beta_m(V)=\frac{0.3\cdot(V-53)}{1-exp(0.2\cdot(V-53))} \ 
\end{equation}

\begin{equation}
\alpha_h(V)=\frac{0.23}{exp((V-3)/20)}, \ \  \beta_h(V)=\frac{3.33}{1+exp(-0.1\cdot(V-55.5))}\ \ \ 
\end{equation}

The fast delayed rectifier potassium current, \textit{I\textsubscript{Kdr}} is given by:
\begin{equation}
I_{Kdr} = g_{max, Kdr} \cdot n^4 \cdot (V-E_K)
\end{equation}

\begin{equation}
\alpha_{n}=\frac{-0.07\cdot(V-47)}{1-exp((V-47)/-6)}, \ \  \beta_{n}=0.264\cdot exp((V-22)/4)
\end{equation}

The N-type calcium current, \textit{I}\textit{\textsubscript{CaN}}, is given by

\begin{equation}
I_{CaN}=g_{max, CaN} \cdot c^2 \cdot d \cdot (V-E_{Ca})
\end{equation}

\begin{equation}
\alpha_c(V)=\frac{0.19\cdot(19.88-V)}{ exp(0.1 \cdot(19.88-V))-1}, \  \beta_c(V)=0.046 \cdot exp(-V/20.73)
\end{equation}

\begin{equation}
\alpha_d=1.6\cdot10^{-4}\cdot exp(-V/48.4) , \  \beta_d=\frac 1{1+exp(0.1 \cdot (39-V))}
\end{equation}

The Ca\textsuperscript{2+}-dependent K\textsuperscript{+} (SK) current, \textit{I\textsubscript{AHP}}, is described by

\begin{equation}
I_{AHP} = g_{max, AHP} \cdot q^2 \cdot (V-E_K)
\end{equation}

\begin{equation}
\alpha_q(V)=\frac{0.00246}{exp((12\cdot log_{10}([Ca^{2+}]_{in})+28.48)/-4.5)}
\end{equation} 
\begin{equation}
\beta_q(V)=\frac{0.006}{exp((12\cdot log_{10}([Ca^{2+}]_{in})+60.4)/35)}
\end{equation}

\begin{equation}
\label{eq:CaDynamics}
\frac{d[Ca^{2+}]_{in}}{dt}=B\sum_{T, N, L}
I_{Ca}-\frac{[Ca^{2+}]_{in}-[Ca^{2+}]_0}{\tau_{Ca}}
\end{equation}

where
\begin{equation}
B = 5.2\cdot 10^{-6}/(A \cdot d)
\end{equation}
in units of mol/(C m\textsuperscript{3}) for a shell of
surface area \textit{A} and thickness \textit{d} (0.2 $\mu $m) and  $\tau_{Ca}=10$ ms was the calcium removal rate. [\textit{Ca\textsuperscript{2+}}]\textit{\textsubscript{0}} = 5 $\mu$M was the resting calcium concentration. 
The Ca\textsuperscript{2+} and voltage-dependent K\textsuperscript{+} (BK) current, \textit{I\textsubscript{c}}, is:
\begin{equation}
I_C=g_{max, c} \cdot o \cdot (V-E_K)
\end{equation}

\begin{equation}
\alpha_{o} = \frac{0.28 \cdot[Ca^{2+}]_{in}}{[Ca^{2+}]_{in} + 0.00048	\cdot exp(-1.68 \cdot F \cdot V / (R \cdot T)) }
\end{equation}
\begin{equation}
\beta_{o} = \frac{0.48}{1 + [Ca^{2+}]_{in} / (0.13 \cdot 10^{-6} \cdot exp(-2 \cdot F \cdot V/ (R \cdot T))) )}
\end{equation}

The A-type K\textsuperscript{+} current,
\textit{I}\textit{\textsubscript{A}}, is described by\ \ 
\begin{equation}
I_A=g_{max, A} \cdot a \cdot b  \cdot (V-E_K)
\end{equation}

\begin{equation}
\alpha_a=\frac{0.02\cdot (13.1-V)}{exp(\frac{13.1-V}{10})-1}, \ 
\beta_a=\frac{0.0175  \cdot(V-40.1)}{exp(\frac{V-40.1}{10})-1}
\end{equation}

\begin{equation}
\alpha_b = 0.0016 \cdot exp \Big(\frac{V + 13}{-18} \Big), \  \beta_b=\frac{0.05}{1+exp(\frac{10.1-V}{5})}
\end{equation}

The L-type Ca\textsuperscript{2+} current, \textit{I\textsubscript{CaL}}, is described by


\begin{equation}
I_{CaL}=g_{max, CaL}\cdot s_{\infty }^2\cdot V \cdot
\frac{1-\frac{exp(2 \cdot F \cdot V/ (k \cdot T)) \cdot [Ca^{2+}]_{in}}{[Ca^{2+}]_{0}}} {1-exp(2 \cdot F \cdot V/(k \cdot T))} 
\end{equation}


where \textit{g}\textit{\textsubscript{CaL}} is the maximal conductance, \textit{s}\textsf{\textit{\textsubscript{∞}}}
is the steady-state activation variable, \textit{F} is Faraday’s constant, \textit{T} is the temperature, \textit{k} is
Boltzmann’s constant, [\textit{Ca\textsuperscript{2+}}]\textit{\textsubscript{0}} \ is the equilibrium calcium concentration and [\textit{Ca\textsuperscript{2+}}]\textit{\textsubscript{in}} is
described in equation (\ref{eq:CaDynamics}). The activation variable, \textit{s\textsubscript{∞}}, is then


\begin{equation}
\alpha_s(V) = \frac{15.69\cdot(-V+81.5)}{exp(\frac{-V+81.5}{10})-1} , \ \  
\beta_s(V )= 0.29\cdot exp(-V/10.86)
\end{equation}

\begin{table}[h]
	\caption{ Parameters of pyramidal neurons  }
	\label{table:ca1_pyramidal_cell_parameters}
	\begin{center}
		\begin{tabular}{ |p{2cm} | p{1.25cm} |p{1.25cm} |p{1.25cm} |p{1.25cm} |p{1.25cm} |p{1.25cm} |p{1.25cm} |p{1.25cm} | }
			\hline
			\textbf{Parameter} & \textbf{Soma} & \textbf{Axon} & \textbf{OriProx} & \textbf{OriDist} & \textbf{RadProx} & \textbf{RadMed} & \textbf{RadDist} & \textbf{LM} \\ \hline
			Cm, mF/cm2 & 1 & 1 & 1 & 1 & 1 & 1 & 1 & 1 \\ \hline Rm, Ohm cm2 & 20000 & 20000 & 20000 & 20000 & 20000 & 20000 & 20000 & 20000 \\ \hline  Ra, $\Omega$ cm & 50 & 50 & 50 & 50 & 50 & 50 & 50 & 50 \\ \hline Leak conductance (S/cm2) & 0.0002 & 0.000005 & 0.000005 & 0.000005 & 0.000005 & 0.000005 & 0.000005 & 0.000005 \\ \hline  Sodium conductance (S/cm2) & 0.007 & 0.1 & 0.007 & 0.007 & 0.007 & 0.007 & 0.007 & 0.007 \\ \hline  Delayed Rectifier K+ conductance (S/cm2) & 0.0014 & 0.02 & 0.000868 & 0.000868 & 0.000868 & 0.000868 & 0.000868 & 0.000868 \\ \hline  Proximal A-type K+ conductance (S/cm2) & 0.0025 & --- & 0.0075 & 0.0075 & 0.015 & 0 & 0 & --- \\ \hline  Distal A-type K+ conductance (S/cm2) & --- & --- & 0 & 0 & 0 & 0.03 & 0.045 & 0.049 \\ \hline  M-type K+ conductance (S/cm2) & 0.06 & 0.03 & 0.06 & 0.06 & 0.06 & 0.06 & 0.06 & --- \\ \hline  Ih conductance [S/cm2] & 0.00005 & --- & 0.00005 & 0.0001 & 0.0001 &  0.0002 & 0.00035 & --- \\ \hline  Vhalf,h (mV) & -73 & --- & -81 & -81 & -82 & -81 & -81 & --- \\ \hline  L-type Ca2+ conductance (S/cm2) & 0.0007 & --- & 0.000031635 & 0.000031635 & 0.000031635 & 0.0031635 & 0.0031635 & --- \\ \hline  T-type Ca2+ conductance (S/cm2) & 0.00005 & --- & 0.0001 & 0.0001 & 0.0001 & 0.0001 & 0.0001 & --- \\ \hline  R-type Ca2+ conductance (S/cm2) & 0.0003 & --- & 0.00003 & 0.00003 & 0.00003 & 0.00003 & 0.00003 & --- \\ \hline  Ca2+-dependent sAHP  K+ conductance (S/cm2) & 0.0005 & --- & 0.0005 & 0.0005 & 0.0005 & 0.0005 & 0.0005 & --- \\ \hline  Ca2+-dependent mAHP  K+ conductance (S/cm2) & 0.09075 & --- & 0.033 & 0.033 & 0.033 & 0.033 & 0.0041 & --- \\ \hline  EL (mV) & -70 & -70 & -70 & -70 & -70 & -70 & -70 & -70 \\ \hline  ENa (mV) & 50 & 50 & 50 & 50 & 50 & 50 & 50 & 50 \\ \hline  Eh, (mV) & -10 & --- & -10 & -10 & -10 & -10 & -10 & -10 \\ \hline  ECa, (mV) & 140 & --- & 140 & 140 & 140 & 140 & 140 & --- \\ \hline  EK (mV) & -80 & -80 & -80 & -80 & -80 & -80 & -80 & -80 \\ \hline  
			
		\end{tabular}
	\end{center}
\end{table}

\subsection{PV basket cells}
PV basket cells were described with 17-compartment model, 1 - soma and 16 - dendrites. Balace equation for all compartments:
\begin{eqnarray}
C\frac{dV}{dt}=-I_L-I_{Na}-I_{Kdr}-I_{CaL}-I_{CaN}-I_{KCaB}- \nonumber \\ 
-I_{KCaS}-I_{KA} -I_{syn} + I_{ext}
\end{eqnarray}

Sodium current for spike generation is:
\begin{equation}
\label{eq:Nav}
I_{Na} = g_{max, Na} \cdot m^3 \cdot h \cdot (V - E_{Na})
\end{equation}
\begin{equation}
\alpha_m = \frac{-0.3 \cdot (V + 43)}{exp(-0.2\cdot(V+43)) - 1}
\end{equation}
\begin{equation}
\beta_m = \frac{0.3 \cdot (V + 15)}{exp(0.2\cdot(V+15)) - 1}
\end{equation}
\begin{equation}
\alpha_h = \frac{0.23}{exp(0.05\cdot(V+65))}
\end{equation}
\begin{equation}
\beta_h = \frac{3.33}{exp(-0.1\cdot(V+12.5)) + 1}
\end{equation}

Delayed rectification potassium current is
\begin{equation}
\label{eq:Kdrfast}
I_{Kdr} = g_{max, Kdr} \cdot n^4 \cdot (V - E_K)
\end{equation}
\begin{equation}
\alpha_n = \frac{-0.07 \cdot(V + 18)}{exp(\frac{V + 18}{-6}) - 1}
\end{equation}

\begin{equation}
\beta_n = 0.264 \cdot exp \Big( \frac{V + 43}{40} \Big)
\end{equation}

L-type of calcium current is: 
\begin{equation}
\label{eq:CavL}
I_{CaL} = g_{max, CaL} \cdot m^2 \cdot h \cdot ghk(V, [Ca_{2+}]_{in}, [Ca_{2+}]_{out} )
\end{equation}

\begin{equation}
h = \frac{0.001}{0.001 +[Ca^{2+}]_{in} }
\end{equation}

\begin{equation}
\alpha_{m} = \frac{15.69 \cdot (81.5-V)}{exp(0.1(81.5-V)) - 1}
\end{equation}

\begin{equation}
\beta_{m} = 0.29 \cdot exp \Big(\frac{-V}{10.86}\Big)
\end{equation}

\begin{equation}
ghk = -x \cdot \Big(1 - \Big(\frac{[Ca_{2+}]_{in}}{[Ca_{2+}]_{out}} \Big) \cdot exp \Big(\frac{V}{x}\Big) \Big) \frac{V}{x \cdot exp(V/x) -1 }
\end{equation}
where
\begin{equation}
x = 0.04259 \cdot T
\end{equation}
Calcium current N-type is given by following equation: 
\begin{equation}
\label{eq:CavN}
I_{CaN} = g_{max, CaN} \cdot c^2 \cdot d \cdot (V - E_{Ca})
\end{equation}

\begin{equation}
\alpha_{c} = \frac{-0.19 \cdot (V - 19.88)}{exp(0.1(V - 19.88)) - 1}
\end{equation}

\begin{equation}
\beta_{c} = 0.046 \cdot exp \Big(\frac{-V}{20.73}\Big)
\end{equation}

\begin{equation}
\alpha_{d} = 0.00016 \cdot exp \Big(\frac{-V}{48.4}\Big)
\end{equation}

\begin{equation}
\beta_{d} = \frac{1}{exp(0.1 \cdot (39 - V)) + 1}
\end{equation}


Calcium-dependent potassium current B:
\begin{equation}
I_{KCaB} = g_{max, KCaB} \cdot n \cdot (V - E_K)
\label{eq:KvCaB}
\end{equation}
\begin{equation}
\alpha_n = \frac{0.28 \cdot [Ca^{2+}]_{in} }{ [Ca^{2+}]_{in} + 0.00048 \cdot exp(\frac{-1.68 \cdot F \cdot V}{R \cdot T})  } 
\end{equation}
\begin{equation}
\beta_n = \frac{0.48}{1 + 130000 \cdot [Ca^{2+}]_{in} \cdot exp(\frac{2 \cdot F \cdot V}{R \cdot T})}
\end{equation}

Calcium-dependent potassium current S:
\begin{equation}
\label{eq:KvCaS}
I_{KCaS} = g_{max, KCaS} \cdot q^2 \cdot (V - E_K)
\end{equation}
\begin{equation}
\alpha_q = q_{10} \cdot 15 \cdot ([Ca^{2+}]_{in})^2 , \  \beta_q = q_{10} \cdot 0.00025
\end{equation}
\begin{equation}
q10 = 3^{0.1\cdot (T - 307) } 
\end{equation}

A-type of potassium current is:
\begin{equation}
\label{eq:KvA}
I_{KA} = g_{max, KA} \cdot n \cdot l \cdot (V - E_K)
\end{equation}
\begin{equation}
n_{\infty} = \frac{1}{1 + exp(-21 \cdot (V + 33.6)/T)}
\end{equation}
\begin{equation}
\tau_n = \frac{exp(-21 \cdot (V + 33.6)/T) }
{q_{10} \cdot 0.02 \cdot (1 + exp(-21 \cdot (V + 33.6)/T))}
\end{equation}

\begin{equation}
l_{\infty} = \frac{1}{1 + exp(46.41 \cdot  (V + 83)/T)}
\end{equation}

\begin{equation}
\tau_l = \frac{exp(46.41 \cdot  (V + 83)/T) }
{q_{10} \cdot 0.08 \cdot (1 + exp(46.41 \cdot  (V + 83)/T))}
\end{equation}

\begin{equation}
q10 = 3^{0.1\cdot (T - 303) } 
\end{equation}


\subsection{CCK basket cells}
Холецистокенин-содержащие корзинчатые нейроны так же как и парвальбумин-содержащие описывались моделью из 17 компартментов, 1 - тело и 16 компартментов дендриты. Потенциал всех компарментов описывался следующим уравнением:
\begin{eqnarray}
C\frac{dV}{dt}=-I_L-I_{Na}-I_{Kdr}-I_{CaL}-I_{CaN}-I_{HCN}-I_{KCaS}- \nonumber \\
-I_{KA}-I_{KCaB}-I_{KGroup}-I_{syn} + I_{ext}
\end{eqnarray}

Sodium current is:
\begin{equation}
\label{eq:Navcck}
I_{Na} = g_{max, Na} \cdot m^3 \cdot h \cdot s \cdot (V - E_{Na})
\end{equation}
\begin{equation}
\alpha_m = \frac{-0.5 \cdot (V + 42)}{exp(-0.2\cdot(V+42)) - 1}
\end{equation}
\begin{equation}
\beta_m = \frac{0.3 \cdot (V + 13)}{exp(0.2\cdot(V+13)) - 1}
\end{equation}
\begin{equation}
\alpha_h = \frac{0.6}{exp(0.05\cdot(V+65))}
\end{equation}
\begin{equation}
\beta_h = \frac{1.31}{exp(-0.1\cdot(V+12.5)) + 1}
\end{equation}
\begin{equation}
\alpha_s = \frac{0.003}{exp( \frac{V+45}{6})}
\end{equation}
\begin{equation}
\beta_s = \frac{0.005}{exp(-0.05\cdot(V+35))}
\end{equation}

H-current is:
\begin{equation}
\label{eq:HCN}
I_{H} = g_{max, H} \cdot H^2 \cdot (V - E_{H})
\end{equation}

\begin{equation}
\tau_{H} = \frac{1}{q_{10}} \cdot \Bigg(120 + \frac{129.5}{1 + exp(1.2 \ (V + 59.3))} \Bigg)
\end{equation}

\begin{equation}
H_{\infty} =  \frac{1}{1 + exp(0.1 \ (V + 91))}
\end{equation}

\begin{equation}
q10 = 3^{0.1\cdot (T - 307) } 
\end{equation}

Potassium current:
\begin{equation}
\label{eq:KvGroup}
I_{KGroup} = g_{max, KGroup} \cdot n \cdot (V - E_K)
\end{equation}
\begin{equation}
\alpha_n(V) = \frac{-0.0189324 \cdot (V - 4.18371) }{exp(-0.15562\cdot (V - 4.18371)) - 1}
\end{equation}

\begin{equation}
\beta_n(V) = 0.015857 \cdot exp \Big(\frac{-V}{25.4834}\Big)
\end{equation}


$ I_{Kdr}, I_{CaL}, I_{CaN},I_{KCaS}, I_{KCaB}, I_{KA}$ were same as PV basket cells and described equations (\ref{eq:Kdrfast}, \ref{eq:CavL}, \ref{eq:CavN}, \ref{eq:KvCaS}, \ref{eq:KvCaB}, \ref{eq:KvA}) respectivly.

\subsection{OLM cells}
OLM нейроны моделировались с помощью 4 компартментов - 1 тело, 1 аксон и 2 дендритных компартмента. Уравнение потенциала для тела нейрона:
\begin{eqnarray}
C\frac{dV_s}{dt} = -I_L - I_{Na} - I_{Kdr} - I_{H} - I_{KA}-
 \nonumber \\
-I_{syn} + I_{ext}
\end{eqnarray}
Уравнение для дендритов
\begin{eqnarray}
C\frac{dV_d}{dt} = -I_L - I_{Na} - I_{Kdr} - I_{KA}-
\nonumber \\
-I_{syn} + I_{ext}
\end{eqnarray}
Уравнение для аксона
\begin{eqnarray}
C\frac{dV_a}{dt} = -I_L - I_{Na} - I_{Kdr} - I_{syn}
\end{eqnarray}
Натриевые токи, калиевые токи выпрямления моделировались также как и у PV корзинчатых нейронов, уравнения (\ref{eq:Nav}) и (\ref{eq:Kdrfast}) соответственно.
H-каналы:
\begin{equation}
\label{eq:HCNolm}
I_{H} = g_{max, H} \cdot H \cdot (V - E_{H})
\end{equation}

\begin{equation}
H_{\infty} =  \frac{1}{1 + exp(0.98 \cdot(V + 84.1))}
\end{equation}

\begin{equation}
\tau_{H} = 100 + \frac{1}{exp(-(17.9+0.116\cdot V)) + exp(0.09 \cdot V-1.84)   }
\end{equation}

Калиевый ток A-типа:
\begin{equation}
\label{eq:KvAolm}
I_{KA} = g_{max, KA} \cdot a \cdot b \cdot (V - E_K)
\end{equation}
\begin{equation}
a_{\infty} = \frac{1}{1 + exp(\frac{V + 14}{-16.6})  } \ \ \tau_a = 5
\end{equation}
\begin{equation}
b_{\infty} = \frac{1}{1 + exp(\frac{V + 71}{7.3})  }
\end{equation}
\begin{equation}
\tau_b = \frac{1}{\frac{0.000009}{exp(\frac{V - 26}{18.5})}  + \frac{0.014}{exp(\frac{V +70}{-11}) + 0.2} }
\end{equation}


\subsection{Axo-axonic cells}
Аксо-аксональные интернейроны моделировались с помощью 17 компартментной модели: 1 сома и 16 компартментов для дендритов. Все компартменты описывались одинаковым набором каналов:
\begin{eqnarray}
C\frac{dV}{dt}=-I_L-I_{Na}-I_{Kdr}-I_{CaL}-I_{CaN}- \nonumber \\ 
-I_{KCaS}-I_{KA}-I_{KCaB}-I_{syn} + I_{ext}
\end{eqnarray}

Описание всех токов аналогично тем, что использовались выше для других нейронов, см. уравнения (\ref{eq:Nav}, \ref{eq:Kdrfast}, \ref{eq:CavL}, \ref{eq:CavN}, \ref{eq:KvCaS}, \ref{eq:KvA})

\subsection{Ivy and neurogiaform cells}
Ivy и нейроглиаформные описывались с помощью 17 компартментов: 1 сома и 16 дендритных компартментов. Все компартметны в обоих типах нейронов имели одинаковый набор каналов:
\begin{eqnarray}
C\frac{dV}{dt}=-I_L-I_{Na}-I_{Kdr} - I_{CaL}-I_{CaN}- \nonumber \\
-I_{KCaS} -I_{KA}-I_{KCaB}-I_{syn} + I_{ext}
\end{eqnarray}
Натриевые каналы:
\begin{equation}
\label{eq:Navngf}
I_{Na} = g_{max, Na} \cdot m^3 \cdot h \cdot (V - E_{Na})
\end{equation}
\begin{equation}
\alpha_m = \frac{-0.34133 \cdot (V + 24)}{exp(-0.2\cdot(V+24)) - 1}
\end{equation}
\begin{equation}
\beta_m = \frac{0.28483 \cdot (V -4)}{exp(0.2\cdot(V-4)) - 1}
\end{equation}
\begin{equation}
\alpha_h = \frac{0.29648}{exp(0.05\cdot(V+64.4184))}
\end{equation}
\begin{equation}
\beta_h = \frac{3.0931}{exp(-0.1\cdot(V+12.1463)) + 1}
\end{equation}

Калиевые каналы выпрямления:
\begin{equation}
\label{eq:Kdrfastngf}
I_{Kdr} = g_{max, Kdr} \cdot n^4 \cdot (V - E_K)
\end{equation}
\begin{equation}
\alpha_n = \frac{-0.07(V + 8)}{exp(\frac{V + 8}{-6}) - 1}
\end{equation}

\begin{equation}
\beta_n = 0.264 \cdot exp \Big( \frac{V + 33}{40} \Big)
\end{equation}

Калиевые токи А-типа:
\begin{equation}
I_{KA} = g_{max, KA} \cdot n \cdot l \cdot (V - E_K)
\end{equation}
\begin{equation}
n_{\infty} = \frac{1}{1 + exp(-34.8 \cdot (V + 23.6)/T)}
\end{equation}
\begin{equation}
\tau_n = \frac{exp(-34.8 \cdot (V + 23.6)) }
{0.02 \cdot q_{10} \cdot (1 + exp(-34.8 \cdot (V + 23.6)/T))}
\end{equation}

\begin{equation}
l_{\infty} = \frac{1}{1 + exp \cdot (46.41\cdot (V + 83)/T)}
\end{equation}

\begin{equation}
\tau_l = \frac{exp(46.41\cdot (V + 83)) }
{0.08 \cdot q_{10} \cdot (1 + exp(46.41\cdot (V + 83)/T))}
\end{equation}

\begin{equation}
q10 = 3^{0.1\cdot (T - 303) } 
\end{equation}

Описание всех остальных токов аналогично тем, что использовались выше для других нейронов, см. уравнения (\ref{eq:CavL}, \ref{eq:CavN}, \ref{eq:KvCaS}, \ref{eq:KvCaB})

\subsection{Schaffer collateral associated cells}
SCA cells описывались 17 компартмментами, 1 - сома и 16 дендриты. Все компарменты описывались с помощью следующего уравнения баланса:
\begin{eqnarray}
C\frac{dV_s}{dt}=-I_L-I_{Na}-I_{Kdr}-I_{CaL}-I_{CaN}-I_{H}-I_{KCaS}- \nonumber \\ -I_{KA}-I_{KCaB}-I_{KGroup}-I_{syn} + I_{ext}
\end{eqnarray}

Натриевый ток, $I_{KGroup}$ and $I_{H}$ описывались также как и для CCK basket cells - equation  (\ref{eq:Navcck}, \ref{eq:KvGroup}, \ref{eq:HCN}), уравнения для других каналов были аналогичны PV basket cells  (\ref{eq:CavL}, \ref{eq:CavN}, \ref{eq:KvCaS}, \ref{eq:KvA})




\section{Description and numerical ingration of gate variables}
All gate variables were described with classical equation: 
\begin{equation}
\frac{dx}{dt} = \frac{x_{\infty}(V) - x}{\tau_x(V)}
\end{equation}
The equations for $x_{\infty}(V)$ and $\tau_x(V)$ are given in the corresponding sections in in explicit form or they can be given in form of functions  $\alpha(V)$ and $\beta(V)$
\begin{equation}
\tau_x(V) =  \frac{1}{\alpha(V) + \beta(V)} \ \ \ 
x_{\infty}(V) = \alpha(V) \cdot \tau_x(V)
\end{equation}
Numerical scheme:
\begin{equation}
x_{t + \Delta t} = x_t+\Big(1 + exp \Big(-\frac{\Delta t}{\tau_x} \Big) \Big)\cdot (x_{\infty}-x_t) 
\end{equation}
where $\Delta t = 0.1 ms$ is integration step.


\end{document}


