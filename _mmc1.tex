\documentclass[a4paper]{article}
\usepackage{amsmath}
\title{Supporting Online Material}
\author{}
\begin{document}
\subsubsection{CA1 Pyramidal cell}


\begin{equation}
C\frac{dV_s}{dt}=-I_L-I_{Na}-I_{kdr}-I_A-I_M-I_h-I_{sAHP}-I_{mAHP}-I_{CaL}-I_{CaT}-I_{CaR}-I_{buff}-I_{syn}
\end{equation}

\begin{equation}
C\frac{dV_a}{dt}=-I_L-I_{Na}-I_{kdr}-I_M-I_{syn}
\end{equation}



\begin{equation}
C\frac{dV_{rad,ori}}{dt} =-I_L-I_{Na}-I_{kdr}-I_A-I_M-I_h-I_{sAHP}-I_{mAHP}-I_{CaL}-I_{CaT}-I_{CaR}-I_{buff}-I_{syn}
\end{equation}

\begin{equation}
C\frac{dV_{LM}}{dt}=-I_L-I_{Na}-I_{kdr}-I_A-I_{syn}
\end{equation}


where \textit{I}\textit{\textsubscript{L}} is the leak current, \textit{I}\textit{\textsubscript{Na}} is the fast sodium
current, \textit{I}\textit{\textsubscript{kdr}} is the delayed rectifier potassium current,
\textit{I}\textit{\textsubscript{A}} is the A-type K\textsuperscript{+} current, \textit{I}\textit{\textsubscript{M}}
is the M-type K\textsuperscript{+} current, \textit{I}\textit{\textsubscript{h}} is a hyperpolarizing h-type current,
\textit{I}\textit{\textsubscript{CaL}}, \textit{I}\textit{\textsubscript{CaT}} and
\textit{I}\textit{\textsubscript{CaR}} are the L-, T- and R-type Ca\textsuperscript{2+} currents, respectively,
\textit{I}\textit{\textsubscript{sAH}}\textsubscript{P} and \textit{I}\textit{\textsubscript{mAH}}\textsubscript{P} are
slow and medium Ca\textsuperscript{2+} activated K\textsuperscript{+} currents,\textit{ I}\textit{\textsubscript{buff}}
is a calcium pump/buffering mechanism and \textit{I}\textit{\textsubscript{syn}} is the synaptic current. The
conductance and reversal potentials values for all ionic currents are listed in Table 7.


The sodium current is described by:
\bigskip
\begin{equation}
I_{Na}= g_{Na}\cdot m^2\cdot h\cdot s\cdot (V-E_{Na})
\end{equation}


\bigskip


where an additional variable {s’ } is introduced to account for dendritic location‑dependent slow attenuation of
the sodium current (Poirazzi et al., 2003a, b). Activation and inactivation kinetics for ${I_{Na}}$ are given by:


\begin{equation}
m_{t + dt} = m_t+(1+e^{-\frac{dt}{τ_m}})\cdot
(m_{\infty}-m_t) \ , \ \ \  m_{\infty}=\frac {1}{1+e^{(-\frac{V+40} {3})}}
\end{equation}


\begin{equation}
h_{t+dt}=h_t+(1-e^{-\frac{dt}{τ_h}})\cdot
(h_{\infty}-h_t) \ ,\\  h_{\infty}=\frac {1}{1+e^{(\frac{V+45}{3})}}
\end{equation}

\begin{equation}
s_{t+dt}=s_t+(1+e^{-\frac{dt}{\tau_s})}\cdot
(s_{\infty}-s_t) ,\ 
s_{\infty}=\frac{1+Na_{att}\cdot e^{(\frac{V+60}
2)}}{1+e^{(\frac{V+60}{2})}}
\end{equation}


with \textit{dt}= 0.1 ms and time constants τ\textsubscript{m}= 0.05 ms, τ\textsubscript{h} = 0.5
ms, and


\begin{equation}
τ_s=\frac{0.00333\cdot  e^{0.0024 \cdot (V+60)\cdot
Q(degC)}}{1+e^{0.0012 \cdot (V+60)\cdot Q(degC)}}
\end{equation}

The function \textit{Q}(\textit{degC})~is given by:
\bigskip

\begin{equation}
Q(degC)= \frac{F}{ R \cdot (T+degC) }
\end{equation}
\bigskip


where \textit{R }=\textit{ }8.315\textit{ joule/degC}, \textit{F }=\textit{ }9.648\textit{ }10\textsuperscript{4}Coul,
\textit{T }=\textit{273.16}~in degrees Kelvin and \textit{degC}~is the temperature in degrees celsius.~variable represents the degree of sodium current
attenuation and varies linearly from soma to distal trunk
Na\textsubscript{att} ε~[0, 1]: 1 -maximum 0 - zero attenuation). The delayed rectifier current is given by:


\begin{equation}
I_{Kdr} = g_{Kdr} \cdot m^2 \cdot (V-E_K)
\end{equation}

\begin{equation}
m_{t+dt}=m_t+(1-e^{-dt/2.2})\cdot m_{\infty}-m_t) ,\ m_{\infty}=\frac{1}{1+e^{(-\frac{V+42}{2})}}
\end{equation}

The sodium and delayed rectifier channel properties are slightly different in the soma, axis and dendritic arbor. To fit
experimental data regarding the backpropagation of spike trains, soma and axon compartments have a lower threshold for
Na\textsuperscript{+} spike initiation (-57 mV) than dendritic ones -50 mV. Thus, the $m_{\infty}$ and $h_{\infty}$
somatic/axonic HH channel kinetics as well as the time constants for both 
$I_{Na}^{sa}$
and 
$I_{Kdr}^{sa}$,
are modified as follows. For the sodium


\begin{equation}
m_{\infty}^{sa}=\frac {1}{1+e^{(-\frac{V+44}{3})}} ,\ \ 
h_{\infty}^{sa}=\frac {1}{1+e^{(\frac{V+49}{3.5})}}
\end{equation}

while for the potassium delayed rectifier

\begin{equation}
m_{\infty}^{sa}=\frac {1}{1+e^{(-\frac{V+46.3}{3})}}
\end{equation}

The somatic time constant for somatic/axonic \textit{Na}\textit{\textsuperscript{+}}~channel activation is kept the
same τ\textsubscript{m} = 0.05 ms while for inactivation is set to τ\textsubscript{h}= 1 ms. The τ-value for the delayed rectifier channel activation is set to $τ_{m}= 3.5 ms$. In all of the
following equations, τ-values~are given in ms.



The fast inactivating A-type K\textsuperscript{+ }current is described by


\begin{equation}
I_Á = g_A\cdot n_A\cdot l\cdot (V-E_K)
\end{equation}

\begin{equation}
n_A(t+1)=n_A(t)+(n_{A_{\infty }}-n_A(t))\cdot (1-e^{-dt/τ_n}),\ \ where 
τ_n=0.2 ms
\end{equation}

\begin{equation}
n_{A_{\infty }}=\frac{α_{n_A}}{α_{n_A}+β_{n_A}}
\end{equation}

\begin{equation}
α_{n_A}=\frac{-0.01(V+21.3)}{e^{-(V+21.3)/35}-1} ,\ \  β_{n_A}=\frac{0.01(V+21.3)}{e^{(V+21.3)/35}-1}
\end{equation}

\begin{equation}
l(t+1)=l(t)+(l_{\infty }-l(t))\cdot (1-e^{-dt/τ_l})
\end{equation}


\begin{equation}
l_{\infty }=\frac{α_l}{α_l+β_l}
\end{equation}

\begin{equation}
α_l=\frac{-0.01(V+58)}{e^{(V+58)/8.2}-1} ,\ \  β_l=\frac{0.01(V+58)}{e^{-(V+58)/8.2}-1}
\end{equation}

where
\begin{equation}
τ_l=5+2.6(V+20)/10\ , if V > 20 mV and  τ_l=5, elsewhere.
\end{equation}

The hyperpolarizing h-current is given by

\begin{equation}
I_h=g_h\cdot tt \cdot (V-E_h)
\end{equation}

\begin{equation}
\frac{dtt}{dt}=\frac{tt_{\infty}-tt}{τ_tt}
\end{equation}

\begin{equation}
tt_{\infty }=\frac {1}{1+e^{-(V-V_{half})/k_l}} \ , \ 
τ_{tt}=\frac{e^{0.0378\cdot ς\cdot gmt}\cdot
(V-V_halft}{qtl \cdot q10^{(T-33)/10}\cdot a0t \cdot
(1+a_{tt})}
\end{equation}

\begin{equation}
a_{tt}=e^{0.00378\cdot ς\cdot (V-V_{halft})}
\end{equation}

{where }{\textsf{\textit{ζ}}}{, }{\textit{gmt}}{, }{\textit{q10}}{ and }{\textit{qtl}}{ are 2.2, 0.4, 4.5 and 1,
respectively, }{\textit{a0t}}{ is 0.0111 1/ms,}



{\textit{V}}{\textit{\textsubscript{halft}}}{ = -75mV and }{\textit{k}}{\textit{\textsubscript{l}}}{ = -8.}



The slowly activating voltage‑dependent potassium current, \textit{I}\textit{\textsubscript{M}},~is given by the
equations:


\begin{equation}
I_m=10^{-4}\cdot T_{adj}(degC)\cdot g_m\cdot m\cdot
(V-E_K)
\end{equation}



\begin{equation}
m_{t+dt}=m_t+(1-e^{-\frac{dt \cdot T_adj(degC)}{τ}}\cdot (\frac{α(V)}{(α(V)+β(V)}-m_t), \ 
 T_{adj}(degC)=2.3^{degC-23)/10}
\end{equation}

\begin{equation}
α(V) = 10^{-3}\cdot \frac{(V+30)}{(1-e^{-(V+30)/9})} \ ,         β(V) = -10^{-3}\cdot \frac{(V+30)}{(1-e^{(V+30)/9})} \ , 
τ=\frac 1{α(V)+β(V)}
\end{equation}


The slow after-hyperpolarizing current, is given by:


\begin{equation}
I_{sAHP}= g_{sAHP}\cdot m^3\cdot
(V-E_K)
\end{equation}

\begin{equation}
\frac{dm}{dt}=\frac{\frac{Cac}{(1+Cac)}-m}{τ}
\end{equation}

%\begin{equation}
%τ=\text{max}(\frac 1{0.003(1/\normalsubformula{\text{ms}})\cdot (1+\normalsubformula{\text{Cac}})\cdot
%3^{(\text{deg}C-22)/10}},0.5
%\end{equation}

where 
$Cac=([ca]_{in}/0.025)^2$.



The medium after-hyperpolarizing current, I\textsubscript{mAHP} (Moczydlowski and Latorre, 1983), is given by:


\begin{equation}
I_{mAHP}= g_{mAHP}\cdot m\cdot
(V-E_K)
\end{equation}

\begin{equation}
m_{t+dt}=m_t+(1+e^{-\frac{dt}{τ_m}})\cdot (\frac{α_m(V)}{τ_m}-m_t)
\end{equation}

\begin{equation}
α_m(V)=\frac{0.48}{1+\frac{0.18}{[Ca]_{in}}\cdot
e^{-1.68\cdot V\cdot Q(degC)}}
\end{equation}

\begin{equation}
β_m(V)=\frac{0.28}{1+\frac{[ca]_{in}}{0.011\cdot
e^{-2\cdot V\cdot Q(degC)}}}
\end{equation}

\begin{equation}
τ_m=\frac {1}{α_m(V)+β_m(V)}
\end{equation}

The somatic high-voltage activated (HVA) L-type Ca\textsuperscript{2+} current is given by


\begin{equation}
I_{CaL}^s= g_{CaL}^s\cdot m\cdot
\frac{0.001 }{0.001 + [ca]_{in}\cdot
ghk(V, [ca]_{in}, [Ca]_{out})}
\end{equation}

\begin{equation}
α_m(V)=-0.055\cdot \frac{(V+27.01)}{e^{-(V+27.01)/3.8}-1} , \  β_m(V)=0.94\cdot e^{-(V+63.01)/17}
\end{equation}

\begin{equation}
τ_m=\frac {1}{5\cdot (α_m(V)+β_m(V))}
\end{equation}

whereas the dendritic L-type calcium channels have different kinetics: 


\begin{equation}
I_{CaL}^d= g_{CaL}^d \cdot m^3 \cdot h \cdot
(V-E_{Ca})
\end{equation}

\begin{equation}
α(V)=\frac {1}{1+e^{-(v+37)}},β(V)=\frac {1}{1+e^{(v+41)/0.5}}
\end{equation}


Their time constants are equal to τ\textsubscript{m}\textit{ }=\textit{ }3.6\textit{ ms}~and τ\textsubscript{h}\textit{
}=\textit{ }29\textit{ ms}.



The low-voltage activated (LVA) T-type Ca\textsuperscript{2+} channel kinetics are given by:


\begin{equation}
I_{CaT}= g_{CaT}\cdot m^2\cdot
h\frac{0.001 }{0.001+[ca]_{in}\cdot
ghk(V,[ca]_{in},[ca]_{out})}
\end{equation}

\begin{equation}
ghk(V, [ca]_{in}, [ca]_{out})=-x\cdot
(1 - [ca]_{out}^{[ca]_{in}})\cdot
e^{\frac {V} {x})\cdot f(\frac {V}{x})}
\end{equation}

\begin{equation}
x=\frac{0.0853\cdot (T+degC)}{2}, \\
f(z) = \begin{cases} 1-\frac {z}{2}, & if \ |z|<10^{-4} \\ \frac z{e^z-1}, & \  otherwise \ \end{cases}
\end{equation}

\begin{equation}
m_{t+dt}=m_t+(1+e^{-\frac{dt}{τ_m}})\cdot
(\frac{α_{m(V)}}{α_m(V)+β_m(V)}-m_t)
\end{equation}

\begin{equation}
h_{t+dt}=h_t+(1-e^{-\frac{dt}{τ_h}})\cdot
(\frac{α_h(V)}{α_h(V)+β_h(V)}-h_t)
\end{equation}

\begin{equation}
α_m(V)=-0.196\cdot \frac{(V-19.88)}{e^{-(V-19.88)/10}-1},β_m(V)=0.046\cdot e^{-(V/22.73)}
\end{equation}

\begin{equation}
α_h(V)=0.00016\cdot e^{-(V+57)/19},β_h(V)=\frac 1{e^{-(V-15)/10}+1}
\end{equation}

\begin{equation}
τ_m=\frac {1}{α_m(V)+β_m(V)},τ_h=\frac {1}{0.68\cdot (α_h(V)+β_h(V)}
\end{equation}

where
$[Ca]_{in}$ and $[Ca]_{out}$ are the
internal and external calcium concentrations. The HVA R-type Ca2+ current is described by:


\begin{equation}
I_{CaR}=\bar g_{CaR}\cdot m^3\cdot h\cdot
(V-E_{Ca})
\end{equation}

\begin{equation}
m_{t+dt}=m_t+(1+e^{-\frac{dt}{τ_m}})\cdot (α(V)-m_t)
\end{equation}
\begin{equation}
h_{t+dt}=h_t+(1-e^{-\frac{dt}{τ_h}})\cdot (β(V)-h_t)
\end{equation}

The difference between somatic and dendritic CaR currents lies in the α(\textit{V}), β(\textit{V})~and τ~parameter
values. For the somatic current, τ\textsubscript{m}\textit{ }=\textit{ }100\textit{ ms}~and
τ\textsubscript{h}=5\textit{ ms}~while for the dendritic current τ\textsubscript{m}\textit{ }=\textit{ }50\textit{
ms}~and τ\textsubscript{h}=5\textit{ ms}. The α(\textit{V})~and β(\textit{V})~equations for dendritic CaR channels are:


\begin{equation}
α(V)=\frac 1{1+e^{-(v+48.5)/3}},β(V)=\frac 1{1+e^{(v+53)}}
\end{equation}

while for the somatic CaR channels:


\begin{equation}
α(V)=\frac 1{1+e^{-(v+60)/3}},β(V)=\frac 1{1+e^{(v+62)}}
\end{equation}

Finally, a calcium pump/buffering mechanism is inserted at the cell body and along the apical and basal trunk. The
mechanism is taken from~(Destexhe, Mainen, \& Sejnowski, 1994). The factor for
$Ca_{2+}$ entry was changed from
\textit{f}\textit{\textsubscript{e}}\textit{ }=\textit{ }10,000~to \textit{f}\textit{\textsubscript{e}}\textit{
}=\textit{ }10,000/18~and the rate of calcium removal was made 7 times faster. The kinetic equations are given by:

\begin{equation}
drivechannel = \begin{cases} -f_e \frac{I_{Ca}}{0.2 F}, & if \ drivechannel > 0  \\ 0, & \  otherwise \ \end{cases}
\end{equation}

\begin{equation}
\frac{dCa}{dt}=drivechannel +\frac{10^{-4}-Ca}{7\cdot 200 ms}
\end{equation}




\bigskip

\subsubsection{Bistratified cells}

All compartments obey the following current balance equation, which was adapted from Cutsuridis et al. (2010):


\begin{equation}
C\frac{dV}{dt}=I_{ext}-I_L-I_{Na}-I_{Kdr, fast}-I_A-I_{CaL}-I_{CaN}-I_{AHP}-I_C-I_{syn}
\end{equation}


where \textit{C} is the membrane capacitance, \textit{V} is the membrane potential, \textit{I}\textit{\textsubscript{L}}
is the leak current, \textit{I}\textit{\textsubscript{Na}} is the sodium current,
\textit{I}\textit{\textsubscript{Kdr,fast}} is the fast delayed rectifier K\textsuperscript{+} current,
\textit{I}\textit{\textsubscript{A}} is the A-type K\textsuperscript{+} current, \textit{I}\textit{\textsubscript{CaL}}
is the L-type Ca\textsuperscript{2+} current, \textit{I}\textit{\textsubscript{CaN}} is the N-type
Ca\textsuperscript{2+} current, \textit{I}\textit{\textsubscript{AHP}} is the Ca\textsuperscript{2+}-dependent
K\textsuperscript{+} (SK) current, \textit{I}\textit{\textsubscript{C}} is the Ca\textsuperscript{2+} and
voltage-dependent K\textsuperscript{+} (BK) current and \textit{I}\textit{\textsubscript{syn}} is the synaptic current.
The conductance and reversal potential values of all ionic currents are listed in Table 8. \ \ 



The sodium current and its kinetics are described by,


\begin{equation}
I_{Na}=g_{Na} m^3 h (V-E_{Na})
\end{equation}

\begin{equation}
\frac{dm}{dt}=α_m(1-m)-β_mm, \ 
α_m=\frac{-0.3(V-25)}{(1-e^{(V-25)/-5})}, \ \  β_m=\frac{0.3(V-53)}{(1-e^{(V-53)/5})} \ 
\end{equation}

\begin{equation}
\frac{dh}{dt}=α_h(1-h)-β_hh, \ 
α_h=\frac{0.23}{e^{(V-3)/20}}, \ \  β_h=\frac{3.33}{(1+e^{(V-55.5)/-10})}\ \ \ 
\end{equation}


The fast delayed rectifier K\textsuperscript{+} current, \textit{I}\textit{\textsubscript{Kdr,fast}} is given by\ \ 


\begin{equation}
I_{Kdr, fast}=g_{Kdr, fast}n_f^4(V-E_K)
\end{equation}

\begin{equation}
\frac{dn_f}{dt}=α_{n_f}(1-n_f)-β_{n_f}n_f , 
α_{n_f}=\frac{-0.07(V-47)}{(1-e^{(V-47)/-6})}  β_{n_f}=0.264e^{(v-22)/4}
\end{equation}


The N-type Ca\textsuperscript{2+} current, \textit{I}\textit{\textsubscript{CaN}}, is given by


\begin{equation}
I_{CaN}=g_{CaN}c^2d(V-E_Ca)
\end{equation}

\begin{equation}
\frac{dc}{dt}=α_c(1-c)-β_cc , \ 
α_c=\frac{0.19(19.88-V)}{(e^{(19.88-V)/10}-1)}, \  β_c=0.046e^{-V/20.73}
\end{equation}

\begin{equation}
\frac{dd}{dt} = α_d(1-d)-β_dd , \ α_d=1.6\cdot
10^{-4}e^{-V/48.4} , \  β_d=\frac 1{(1+e^{(39-V)/10})}
\end{equation}


The Ca\textsuperscript{2+}-dependent K\textsuperscript{+} (SK) current, \textit{I}\textit{\textsubscript{AHP}}, is
described by


\begin{equation}
I_{AHP} = g_{AHP} q^2 (V-E_K)
\end{equation}

\begin{equation}
\frac{dq}{dt}=α_q(1-q)-β_qq
\end{equation}

\begin{equation}
α_q=\frac{0.00246}{e^{(12\cdot log_{10}([Ca^{2+}])+28.48)/-4.5}} \ \ ,\ \ 
β_q=\frac{0.006}{e^{(12\cdot log_{10}([Ca^{2+}])+60.4)/35}}
\end{equation}

\begin{equation}
 \frac{d[Ca^{2+}]_i}{dt}=B\sum_{T, N, L}
I_{Ca}-\frac{[Ca^{2+}]_i-[Ca^{2+}]_0}{\tau}
\end{equation}

where
\begin{equation}
B = 5.2\cdot 10^{-6}/Ad
\end{equation}
in units of mol/(C m\textsuperscript{3}) for a shell of
surface area \textit{A} and thickness \textit{d} (0.2 μm) and  $τ=10$;ms was the calcium removal rate. \textit{[Ca}\textit{\textsuperscript{2+}}\textit{]}\textit{\textsubscript{0}} = 5 μM was the resting calcium concentration. 


The Ca\textsuperscript{2+} and voltage-dependent K\textsuperscript{+} (BK) current,
\textit{I}\textit{\textsubscript{c}}, is 


\begin{equation}
I_C=g_co(v-E_K)
\end{equation}


where activation variable, \textit{o}, is described in Migliore et al., (1995). The A-type K\textsuperscript{+} current,
\textit{I}\textit{\textsubscript{A}}, is described by\ \ 


\begin{equation}
I_A=g_A ab (V-E_k)
\end{equation}

\begin{equation}
\frac{da}{dt}=α_a(1-a)-β_aa , 
α_a=\frac{0.02(13.1-V)}{e^{(\frac{13.1-V}{10})}-1}, \ 
β_a=\frac{0.0175(V-40.1)}{e^{(\frac{V-40.1}{10})}-1}
\end{equation}

\begin{equation}
\frac{db}{dt}=α_b(1-b)-β_bb , \ 
α_b=0.0016e^{(\frac{-13-V}{18})}, \  β_b=\frac{0.05}{1+e^{(\frac{10.1-V} 5)}}
\end{equation}

\begin{equation}
The L-type Ca\textsuperscript{2+} current, \textit{I}\textit{\textsubscript{CaL}}, is described by
\end{equation}

\begin{equation}
I_{CaL}=g_{CaL}\cdot s_{\infty }^2\cdot V\cdot
% \frac{1-\frac{[Ca^{2+}]_i}}{[Ca^{2+}]_0}}{e^{2FV/kT}(1-e^{2FV/kT})}
\end{equation}


where \textit{g}\textit{\textsubscript{CaL}} is the maximal conductance, \textit{s}\textsf{\textit{\textsubscript{∞}}}
is the steady-state activation variable, \textit{F} is Faraday’s constant, \textit{T} is the temperature, \textit{k} is
Boltzmann’s constant, [\textit{Ca\textsuperscript{2+}}]\textit{\textsubscript{0}} \ is the
equilibrium calcium concentration and [\textit{Ca}\textit{\textsuperscript{2+}}]\textit{\textsubscript{i}} is
described in equation 49. The activation variable, \textit{s}\textsf{\textit{\textsubscript{∞}}}\textit{,} is then


\begin{equation}
s_{\infty }=\frac{α_s}{α_s+β_s} ,  α_s=\frac{15.69(-V+81.5)}{e^{(\frac{-V+81.5}{10})}-1} ,  
β_s=0.29\cdot e^{-V/10.86}
\end{equation}


\bigskip








\end{document}
